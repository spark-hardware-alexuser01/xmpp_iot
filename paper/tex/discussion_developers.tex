Having different protocols for setting a node`s values and getting a node`s
values can prove to be frustrating for developers. The Control XEP seems to be
the perfect protocol to implement a read request in. Instead, the developer is
forced to support the Sensor Data XEP in order to support readout of meter
values.

Worse yet, the protocols have differences in their view of data types. Sensor
Data has a numeric data type, which is a number with a precision integer and
unit. Control has no concept of the numeric data type, and so the programmer is
forced to implement a default conversion in the sensor if they want the client
to be able to set the numeric data type.

The worst asset to this split up of Control and Sensor data is the
`controllable' flag, which can cause confusion for a user as well, outlined in
the user section below. However, the controllable flag could be turned into a
positive if it were mandatory.

Our solution to the problem of splitting up data fields is that upon giving a
node a data field, the protocol should mandate that the user specify whether the
field is controllable or not through the Sensor Data protocol. That way, Sensor
Data would know the difference between the control parameters (settable) and
fields (only viewable). Essentially, all Sensor Data would need from there is a
set method and Control would cease to serve a purpose.
