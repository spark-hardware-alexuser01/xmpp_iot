In order for a protocol to support IoT, it must be scalable, secure and simple. Scalability is required due to the large number of nodes involved in connecting all things of the world. A thing can be anything from a light in a house to a vehicle to a trash can. Security is important due to the impact IoT will have on the accessibility of important objects. Simplicity and interoperability come hand in hand due to the diversity of the things involved in IoT.\n

XMPP performs well under a heavy load as exemplified by the observation of public XMPP servers such as MongooseIM. MongooseIM is an XMPP server which performed load tests to find the max number of concurrent users. After 400,000 users were online, memory usage was at 66 percent. It was predicted that 500-600 thousand online users all sending messages would be possible [REFERENCE 5]. Scalability one of XMPP’s strong suits.\n

XMPP supports client authentication and authorization through its XMPP Extension Protocols (XEPs), specifically in XEP-0324: Provisioning. This ensures that the server is properly authenticated and any server certificate properly validated. Devices can be set to only accept control commands from authenticated parties. Security is a focus for XMPP community [REFERENCE NEEDED], which is a necessity[REFERENCE NEEDED] for implementing IoT.\n
