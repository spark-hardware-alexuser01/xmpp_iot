In order for a protocol to support IoT, it must be scalable, secure, simple, 
and interoperable[g]. Scalability is required due to the large number of nodes 
involved in connecting all things of the world[h]. Security is important due 
to the impact IoT will have on accessibility[i] of important objects. 
Simplicity and interoperability come hand in hand due to the diversity of the 
things involved in IoT.

XMPP has focused and continues to focus on scalability and security. 
[REFERENCE NEEDED]

XMPP has proven that it can handle many concurrent users at any time through 
the observation of public XMPP servers such as jabber.ru.[j] Jabber.ru handles 
between 10 to 16 thousand concurrent online users on a daily basis[k] 
[REFERENCE 5]. [LARGER EXAMPLE ] [l]Scalability is not an issue for XMPP[m].

XMPP supports encryption, client authentication, and client authorization 
through it's [n]XMPP Extension Protocols (XEPs), specifically in XEP-0324: 
Provisioning[o][p]. This can ensure[q] that the server is properly 
authenticated and any server certificate properly validated. Devices can be 
set to only accept control commands from authenticated parties. Security is a 
focus[r] for XMPP community [REFERENCE NEEDED], which is a 
necessity[s][REFERENCE NEEDED] for implementing IoT.

XMPP IoT currently has 5 [REFERENCE NEEDED]  XEPs[t] trying to tackle the 
goals of IoT.[u] Control, Sensor, Provisioning, Concentrators, and 
Discovery.[v]
