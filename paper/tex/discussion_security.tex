Approach taken as a protocol should be one of that assume devices, services, and
servers not only exist in an unreliable network but an actively malicious one.
Also it may be arguable that there may be no IoT protocol like IPv4 there is for
the Internet. It may from the beginning need to find a commonality among current
protocols to see what minimum features exist amongst all of them and not only
try to provide the best protocol can for their own purposes but assume that
gateways will be able to exist or devices may need to speak multiple IoT
protocols to function in the world around them. 

Provisioning chain of trust. Strange that a spec doesn`t clarify that a device
or user can send an unfriend request. The provisioning server is the only one
who makes that decision.

Provisioning The protocol currently specifies that if there are multiple
provisioning servers on the that are returning access rights (e.g.  CanRead,
CanAccess, CanControl), a device is supposed to apply a Union operation to these
provisioning server responses. This results in a design of potential fail
opening.

Example then is if a company has multiple of provisioning servers for
redundancy and handling larger networks. It doesn`t take the effort of
compromising all of the provisioning servers to abuse the rights of a
device. If you have control of one provisioning server you can override what
may be the valid access controls of all of the other provisioning servers.

Provisioning Authentication. One weakness in authentication of devices,
apart of the XMPP IoT network, is that they only send requests to a service
requesting tokens, which is then delegated between the service and
provisioning server to update or manage the access and controls. The
authentication relies on the SASL and TLS authentication below it and
doesn`t really provide any new level of authentication on the overly IoT
network. It seems to be unclear if the SASL and TLS are sufficient
authentication for devices on a network.`
