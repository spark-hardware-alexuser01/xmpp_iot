The Control XEP is a protocol outlining how to set the controllable parameters 
of a node [footnote explaining a node]. The main goal of the Control XEP is 
to make implementation of data setting commands as simple as possible, which is 
fitting for IoT due to the large diversity of the objects that will implement 
the control protocol.

XMPP set commands, specifically in the Control XEP, are minimal in implementation. 
The only required fields are the recipient and sender, XML namespace, the name 
and value of the control parameter, and its data type. The XML namespace, line 
3 in Figure 1.1 below, exists to define which protocol is being used to 
exchange data. A sample message from a server to a device is provided below in 
Figure 1.1.//*

\lstinputlisting{stanzas/messageStanza.xml}

If an acknowledgement is needed, an iq (info/query) stanza is used instead for more structured exchange of data [REFERENCE 6]. Iq stanzas require a response to any request.
The boolean element on line 4 is a control parameter type. XEP-0325 offers a set of control parameter types that can be validated through either default or application specific validation guidelines defined by the programmer[Reference 7]. The available list of control parameter data types are boolean, color, date, dateTime, double, duration, int, long, string, and time. Since XEP-0325 specifies the control parameter data types, data validation has default behavior through XEP-0122 [Reference 7]. If a parameter does not fall under any of the defined data types, the string data type is the default data type. For example, if a node is a fan that has three modes: low, medium, and high, it would carry a string value with data validation that ensures that the value set is the string:  ``low'',  ``medium'' or  ``high''.

Due to data validation and devices expecting certain data structures, errors will likely arise from malformed set requests. XEP-0325 outlines an error message for this case as well as other likely error cases. Error messages include:  ``bad request'',  ``forbidden'',  ``not found'',  ``not implemented'', and  ``locked''.  ``Bad request'' will arise due to stanzas that are either malformed or requesting a parameter to be set to a value that the device cannot handle.  ``Forbidden'' is due to the caller lacking privileges to perform the action; this comes hand in hand with XEP 0323- Provisioning which will be discussed later.  ``Not found'' will arise if an item, parameter, or data source could not be found.  ``Not implemented'' is in place for a device that knows of the control parameter but has no means of changing it.  ``Locked'' is simply due to an item being locked by another process, for whatever purpose.
For ease of integration, XEP-0325 provides a means of getting all available control parameters of a node with the getForm command. getForm also communicates the data validation rules set by the device, specifically the data types and ranges where appropriate. This convention allows the client to receive all necessary info to begin properly setting controllable parameters. Another convention is that parameters must be ordered in a way so that when set in that order using the typed commands, the corresponding control actions can be successfully executed. For example, a car would not allow the set of a parking brake unless there was first a call to turn off the engine. 
Since a node could have a large set of control parameters, XMPP provides a means of grouping these parameters together with the XData-Layout [REFERENCE] extension. In Figure 1.2 below, XData-Layout 'pages' are used to group together different control actions. Specifically, the main switch of a spotlight is grouped in one page, and the direction in which it is shining is grouped in another. The first page is labeled ‘output’, on lines 18 - 20 which controls if the spotlight is on or off. The other page is labeled ‘direction’, on lines 21 - 24 which controls the x and y-axis of the spotlight.
For control actions that require multiple control parameters, the protocol suggests the usage of the parameterGroup data form, exemplified on line 40 in Figure 1.2. parameterGroup exists to convey that a group of parameters should be set at one time, much like the x and y-axis of the spotlight. parameterGroup is similar to a Java Inherited Parent in that a field may only belong to a maximum of one parameterGroup. 


